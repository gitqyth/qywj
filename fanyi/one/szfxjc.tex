\documentclass{ctexart}

\usepackage{ctex}
\usepackage{amsmath}
\usepackage{xltxtra}
\usepackage{xcolor}
\usepackage{hyperref}
\usepackage[a4paper,left=1.25in,right=1.25in,top=1in,bottom=1in]{geometry}

\title{Notes on Numerical Analysis \\ 数值分析}

\author{Qinghai Zhang}

\begin{document}

\maketitle
\tableofcontents
\addcontentsline{toc}{section}{Solving Nonlinear Equations 求解非线性方程组}
\addcontentsline{toc}{section}{Polynomial Interpolation 多项式插值}
\addcontentsline{toc}{section}{Splines 样条}
\addcontentsline{toc}{section}{Computer Arithmetic 计算机算法}
\addcontentsline{toc}{section}{Approximation 近似值}
\addcontentsline{toc}{section}{Numerical Integration and Differentiation 数值积分与微分}
\addcontentsline{toc}{section}{附录}

\section{Solving Nonlinear Equations 求解非线性方程组}
\subsection{The bisection method 二分法}
二分法通过重复地将间隔减小到根存在的半间隔来求连续函数 f: R $\rightarrow$ R.

\fbox{\begin{minipage}{12cm}
    \textbf{Input}: $f:[a,b] \rightarrow R, a \in R, b \in R, M \in N^+, \delta \in R^+, \epsilon \in R^+$
    
    \textbf{Preconditions(先决条件)}: $f \in C[a,b], sgn(f(a)) \ne sgn(f(b))$
    
    \textbf{Output}: c, h, k
    
    \textbf{Postconditions(后置条件)}: |f(c)| < $\epsilon$ or |h| < $\delta$ or k = M
    
\end{minipage}}
\begin{verbatim}
1 u <- f(a)
2 v <- f(b)
3 for k = 1:M do
4     h <- b-a
5     c <- a+h/2
6     w <- f(c)
7     if |h| < delta or |w| < epsilon Then
8        break
9     else if sgn(w) =! sgn(u) then
10         b <- c  v <- w
11         else
12         a <- c  u <- w
13    end
14 end
\end{verbatim}

\subsection{The signature of an algorithm 算法的签名}
\textbf{Definition 1.2. 算法是一个逐步的过程,它将一组值作为输入,并生成一组值,作为输出。}

(An algorithm is a step-by-step procedure that takes some set of values as its input and produces some set of values as its output.)

\textbf{定义 1.3. 先决条件是在执行算法之前保持输入的条件。}

(A precondition is a condition that holds for the input prior to the execution of an algorithm.)

\textbf{定义 1.4. 后置条件是在执行算法后对输出有效的条件}

(A postcondition is a condition that holds for the output after the execution of an algorithm.)

\textbf{定义 1.5. 算法的签名包括其输入、输出、先决条件、后置条件以及如何处理违反先决条件的输入参数。}

(The signature of an algorithm consists of its input, output, preconditions, postconditions, and how input parameters violating preconditions are handled.)

\subsection{Proof of correctness and simplification of algorithms 算法的正确性证明和简化}
\textbf{定义 1.6. 不变量是在算法执行期间保持不变的条件。}

(An invariant is a condition that holds during the execution of an algorithm.)

\textbf{定义 1.7. 如果变量是在循环中初始化的,则该变量是临时变量或从循环中派生的变量。如果变量在循环之前初始化,并且其值在不同迭代期间发生变化,则变量对于循环是持久的或主要的。}

(A variable is temporary or derived for a loop if it is initialized inside the loop. A variable is persistent or primary for a loop if it is initialized before the loop and its value changes across different iterations.)

\textbf{定义 1.8. 算法1.1中的不变量是什么?a、b、c、h、u、v、w代表哪些量?其中哪些是主要的?以下哪些变量是临时变量?绘制图片以说明这些变量的寿命。}

(What are the invariants in Algorithm 1.1? Which quantities do a, b, c, h, u, v, w represent? Which of them are primary? Which of these variables are temporary? Draw pictures to illustrate the life spans of these variables.)

\textbf{定义 1.9. 一种简化的二等分算法。}

(A simplified bisection algorithm.)

\fbox{\begin{minipage}{12cm}
    \textbf{Input}: $f:[a,b] \rightarrow R, a \in R, b \in R, M \in N^+, \delta \in R^+, \epsilon \in R^+$
    
    \textbf{Preconditions(先决条件)}: $f \in C[a,b], sgn(f(a)) \ne sgn(f(b))$
    
    \textbf{Output}: c, h, k
    
    \textbf{Postconditions(后置条件)}: |f(c)| < $\epsilon$ or |h| < $\delta$ or k = M
    
\end{minipage}}
\begin{verbatim}
1 h <- b-a
2 u <- f(a)
3 for k = 1:M do
4     h <- h/2
5     c <- a+h
6     w <- f(c)
7     if |h| < delta or |w| < epsilon Then
8        break
9     else if sgn(w) = sgn(u) then
10         a <- c
11    end
12 end
\end{verbatim}

\subsection{Q-order convergence Q阶收敛}
\textbf{定义 1.10. (Q-order convergence)收敛序列$\{x_n\}$收敛于L,若:}

\[
\lim_{n \rightarrow \infty} \frac{|x_{n+1} - L|}{|x_n - L|^p} = c > 0; (1.1)
\]

\textbf{常数c称为渐近因子。特别地,当p=1时,$\{x_n\}$具有Q-线性收敛性,而当p=2时,具有Q-二次收敛性。}

(A convergent sequence ${x_n}$ is said to converge to L with Q-order p (p $\ge$ 1),the constant c is called the asymptotic factor. In particular, ${x_n}$ has Q-linear convergence if p = 1 and Q-quadratic convergence if p = 2.)

\textbf{定义 1.11. 迭代序列$\{x_n\}$被称为线性收敛到L,如果:}

\[\exists c \in (0,1),\exists d > 0, s.t. \forall n \in N, |x_n - L| \le c^nd.   (1.2)
\]
\textbf{对于一个收敛到L的序列$\{x_n\}$,其收敛阶数是所有满足$p \in R^+$中最大的那个p。}
\[
\exists c > 0,\exists N \in N s.t. \forall n > N,|x_{n+1}-L| \le c|x_n-L|^p.  (1.3)
\]
\textbf{特别的,如果p=2,则$\{x_n\}$二次收敛}

(A sequence of iterates $\{x_n\}$ is said to converge linearly to L if: For a sequence $\{x_n\}$ that converges to L, its order of convergence is the maximum p $\in$ R^+ satisfying.In particular, $\{x_n\}$ converges quadratically if p = 2.)

\textbf{定义 1.12. (单调序列定理 Monotonic sequence theorem)每个有界单调序列都是收敛的。}

(Every bounded monotonic sequence is convergent.)

\textbf{定义 1.13. (二分法的收敛性 Convergence of the bisection method).对于满足$sgn(f(a_1)) \ne sgn(f(b_2))$的连续函数$f:[a_0,b_0]\rightarrow R$,二分法中的迭代序列线性收敛于渐近因子$\frac{1}{2}$}
\[
\lim_{n \rightarrow \infty} a_n = \lim_{n \rightarrow \infty} b_n = \lim_{n \rightarrow \infty} c_n = \alpha, (1.4)
\]
\[
f(\alpha) = 0, (1.5)
\]
\[
|c_n - \alpha| \le 2^{-(n+1)}(b_0 - a_0), (1.6)
\]

\textbf{其中$[a_n,b_n]$是二分法第n次迭代的间隔,并且$c_n = \frac{1}{2}(a_n+b_n)$。}

(For a continuous function $f : [a_0,b_0] \rightarrow R$ satisfying $sgn(f(a_0)) \ne sgn(f(b_0))$, the sequence of iterates in the bisection method converges linearly with asymptotic factor $\frac{1}{2}$,where $[a_n,b_n]$ is the interval in the nth iteration of the bisection method and $c_n = \frac{1}{2}(a_n + b_n )$.)

\textbf{证明. 由二分法得出:}

\[
a_0 \le a_1 \le a_2 \le \cdots \le b_0,
\]
\[
b_0 \ge b_1 \ge b_2 \ge \cdots \ge a_0,
\]
\[
b_{n+1}-a_{n+1} = \frac{1}{2}(b_n - a_n).
\]

(In the rest of this proof, “lim” is a shorthand for “lim n→∞ .” By Theorem 1.12, both {a n } and {b n } converge. Also, lim(b n − a n ) = lim 2 1 n (b 0 − a 0 ) = 0, hence lim b n = lim a n = α. By the given condition and the algorithm, the invariant f (a n )f (b n ) ≤ 0 always holds. Since f is continuous,
lim f (a n )f (b n ) = f (lim a n )f (lim b n ), then f 2 (α) ≤ 0 implies f (α) = 0. (1.6) is another important invariant that can be proven by induction. Comparing (1.6) to (1.2) yields convergence of the bisection method. Also, the convergence is linear with asymptotic factor as c = 12 .)
\section{Polynomial Interpolation 多项式插值}
\section{Splines 样条}
\section{Computer Arithmetic 计算机算法}
\section{Approximation 近似值}
\section{Numerical Integration and Differentiation 数值积分与微分}
\section{附录}

\end{document}
